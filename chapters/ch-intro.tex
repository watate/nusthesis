\SetPicSubDir{ch-Intro}

\chapter{Introduction}
\vspace{2em}

\section{Brief Description of Project}

This work presents a case study of a learning-based approach to smoothing the movement of a real robotic platform. The robotic platform is operated by an end-to-end neural network which is trained using deep reinforcement learning (DRL) to perform target driven map-less navigation. Previous works~\cite{pfeiffer_reinforced_2018},~\cite{tai_virtual--real_2017},~\cite{xie_learning_2018} have focused more on improving the path-finding or trajectory-generating aspect of map-less navigation while this work focuses on the motion profile planning of map-less navigation. That is, the main idea of this work is to show that smoothing the motion profile generated by the DRL model can be achieved through a learning-based approach. A smooth motion profile is taken here to mean a motion profile that minimizes jerk (time derivative of acceleration).

\section{Project Problem and Scope}
Consider the following histogram of the robot’s angular velocity:

\begin{figure}[!t]
  \centering
  \includegraphics[width=.6\linewidth]{\Pic{png}{paper_angular_velocity_hist}}
  \vspace{\BeforeCaptionVSpace}
  \caption{Histogram shows the angular velocity of a well-trained robot in 100 episodes of simulated map-less navigation in Complex World}
  \label{intro:fig:paper_angular_velocity_hist}
\end{figure}
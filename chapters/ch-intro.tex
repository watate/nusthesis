\SetPicSubDir{ch-Intro}
\label{ch:intro}

\chapter{Introduction}
\vspace{2em}

\section{Project Description}

This work presents a case study of a learning-based approach to smoothing the movement of a real robotic platform. The robotic platform is operated by an end-to-end neural network which is trained using deep reinforcement learning (DRL) to perform target driven map-less navigation. Previous works~\cite{pfeiffer_reinforced_2018},~\cite{tai_virtual--real_2017},~\cite{xie_learning_2018} have focused more on improving the path-finding or trajectory-generating aspect of map-less navigation while this work focuses on the motion profile planning of map-less navigation. That is, the main idea of this work is to test a learning-based approach to motion smoothing and compare it with an approach which hard codes the motion limitations (a non-learning-based approach). A smooth motion profile is taken here to mean a motion profile that has minimal jerk (time derivative of acceleration).

\section{Problem Definition}
\label{sec:intro:problem_def}
Consider the following histogram of the robot’s angular velocity:

\begin{figure}[h]
  \centering
  \includegraphics[width=.9\linewidth]{\Pic{png}{angular_velocity_normal_complex_hist}}
  \vspace{\BeforeCaptionVSpace}
  \caption{Histogram shows the angular velocity of a well-trained robot in 500 episodes of simulated map-less navigation in Complex World~\autoref{appsec:complex_world}}
  \label{intro:fig:paper_angular_velocity_hist}
\end{figure}
\newpage

From ~\autoref{intro:fig:paper_angular_velocity_hist}, we observe that after sufficient training, the DRL model only chooses extreme actions for the robot to take (i.e. maximum velocity at 1.0 and minimum velocity at -1.0). Furthermore, to perform these actions, the robot has to accelerate instantaneously from -1.0 to 1.0 and vice versa. In practice, this is not possible for the robot due to acceleration limits; however, conventional DRL models for map-less navigation \textit{implicitly assume} that arbitrary motion in the configuration space is allowed as long as obstacles are avoided. Moreover, even if this motion profile were possible it would not be ideal as it creates unnecessary jerk and jerk-induced vibrations for the robot. 

~\autoref{intro:fig:paper_angular_jerk_hist} shows that the robot does indeed experience jerk (and at specific magnitudes):

\begin{figure}[h]
  \centering
  \includegraphics[width=.9\linewidth]{\Pic{png}{angular_jerk_normal_complex_hist}}
  \vspace{\BeforeCaptionVSpace}
  \caption{Histogram shows the angular jerk of a well-trained robot in 500 episodes of simulated map-less navigation in Complex World~\autoref{appsec:complex_world}}
  \label{intro:fig:paper_angular_jerk_hist}
\end{figure}
\newpage

Motion control methods that achieve fast motions without residual vibrations are already well-known~\cite{meckl_optimized_1998}. Thus, applying an trapezoidal velocity profile to velocities generated by the DRL model should be able to reduce jerk and vibrations.

However, the main contribution of this report is to demonstrate a case study of a learning-based approach to the same problem. We look at the potential performance impacts, impact to training time, and motion smoothing capabilities that such an approach entails, as well as how it compares to a model with a velocity smoother.

\section{Objectives}
\label{intro:sec:obj}
The objectives of this report are therefore to:
\begin{enumerate}
\item Compare the effects on the motion profile of a learning-based approach to velocity smoothing with a non-learning-based approach to velocity smoothing
\item Compare the success rate for mapless navigation performance of a learning-based approach to velocity smoothing with the performance of a non-learning-based approach to velocity smoothing
\end{enumerate}

\section{Report Outline}
The report will begin with~\autoref{ch:background}, a background in deep reinforcement learning and the deep deterministic policy gradient algorithm. ~\autoref{ch:main} describes the main results of the report in the following sequence:

\begin{enumerate}
\item Introduce smooth motion profiles
\item Present the network architecture used to apply this learning-based approach
\item Discuss the empirical findings of this learning-based approach as well as compare it with a non-learning based approach
\end{enumerate}

In terms of the comparison and analysis, we look at the objectives mentioned in~\autoref{intro:sec:obj}.

Lastly, we conclude our findings and discuss potential future research directions.
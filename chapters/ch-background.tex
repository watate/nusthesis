\SetPicSubDir{ch-Intro}

\chapter{Introduction}
\vspace{2em}

\section{Deep Reinforcement Learning: Background}

Deep reinforcement learning is the combination of reinforcement learning and deep learning. This idea of using neural networks for reinforcement learning, however, is not new and can be dated all the way back to Teasauro's TD-Gammon~\cite{tesauro_temporal_nodate}. In the early 2010s, however, the field of deep learning began to find groundbreaking success, particularly in speech recognition~\cite{dahl_context-dependent_2012} and computer vision~\cite{krizhevsky_imagenet_2017}. This success, combined with the advances in computing power (which drastically reduced training time), allowed the revival in interest of using deep neural networks as universal function approximators for reinforcement learning - leading to deep reinforcement learning.

\section{Overview}

This section includes some tricks extracted from my years' experience of using \LaTeX{}.
Rule of thumb: Doing the right things in the right way can your efforts. 
And I am just the right person telling you this. 

\subsection{Citing, Referring and Hyperlinks}

A book~\cite{BOOK:Gray} is cited.

\begin{figure}[!t]
  \centering
  \includegraphics[width=.6\linewidth]{\Pic{pdf}{vege}}
  \vspace{\BeforeCaptionVSpace}
  \caption{A collection of food.}
  \label{intro:fig:vege}
\end{figure}

An online article~\cite{tpc-c} is cited.

\autoref{intro:fig:vege} demonstrates a collection of vegetables. 
There are broccoli, carrot, pea, onion, snow bean, scallion, etc.
They are healthy food characteristic of high-fiber and low-calorie. 

\begin{align}
  Q_{a}^{b} &= \sum_{x=1}^{\infty} \dfrac{\prod_{y=1}^{x} f(y) - a}{\lim_{z \to x} g(z) + b} + 
               \int_{a}^{b} x^2 dx \label{eq:sample:1}\\
            &= H_{1}(a, b) + H_{2}(a, b) \nonumber\\
            &= \gamma(a, b) \label{eq:sample:2}
\end{align}

\autoref{eq:sample:1}, albeit complex, is equivalent to \autoref{eq:sample:2}.
Note that we do not label the second line of the derivation process.
Given $f(x) = x^{2}$ and $0 < x \leq 1$, we have
\[
  f(1) = 1^{2} = 1
\]
and the result is quite simple. 

You may go to \href{https://github.com/streamjoin/nusthesis}{my GitHub repository} by searching for ``nus latex template'' in Google or accessing the url  \url{https://github.com/streamjoin/nusthesis} directly. 

You may refer to the \href{run:./src/references.bib}{bibliography file} to see its organization.  

\subsection{Hyphenation}

The compound word ``ingredient\zz{}insensitive'', where the hyphen is generated through command \CMD{\zz{}} will be hyphenated for individual words rather than the compound word as a whole. 
In contrast, ``nutrition-oriented'' with normal hyphen will be hyphenated as a whole compound word, which is unlikely to be recognized by \LaTeX\ and therefore no hyphenation will be carried out unless you provide customized hyphenation for it. 

\section{Thesis Synopsis}

The rest of this thesis is organized as follows. 
In \autoref{ch:review}, we conduct a literature review. 
\autoref{ch:rice} provides the study on rice. 
\autoref{ch:noodle} describes noodles. 
We conclude the entire thesis as well as discuss further directions for future research in \autoref{ch:concl}.
